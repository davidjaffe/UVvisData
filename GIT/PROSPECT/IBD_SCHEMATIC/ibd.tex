\documentclass[border={5pt}]{standalone}
%\documentclass[border={5pt},convert={outext={.png},density=600}]{standalone}

\usepackage[usenames,dvipsnames,x11names,svgnames]{xcolor}
\usepackage{maxfl-loc}
\usepackage{xparse}
\usepackage{amsmath}
\usepackage{relsize}

\usepackage{tikz}
\usetikzlibrary{decorations.text,decorations.markings,math,calc,arrows,snakes,shapes}


\renewcommand{\familydefault}{\rmdefault} % use roman (serif) font 
%\renewcommand{\familydefault}{\sfdefault} % use sans serif  font 
\begin{document}

\tikzset{
         ->-/.style          = {decoration={markings,mark=at position #1 with {\arrow{>}}},postaction={decorate}},
         proton/.pic         = {code={\shade[ball color=green] circle (.25);}},
         pastproton/.pic     = {code={\fill[white] circle (.25);
                                      \shade[ball color=green,opacity=0.4] circle (.25);}},
         neutrino/.pic       = {code={\fill[white] circle (.10);
                                      \shade[ball color=blue,opacity=0.6]  circle (.10);}},
         neutron/.pic        = {code={\shade[ball color=blue]  circle (.25);}},
         electron/.pic       = {code={\shade[ball color=red]   circle (.15);}},
         positron/.pic       = {code={\shade[ball color=cyan]  circle (.15);}},
         particlepath/.style = {thick,dashed},
         heavyparticlepath/.style = {thick,dashed,Green!80!black,line width=3.7pt},
         interaction/.style  = {starburst,fill=Gold!80!black,scale=1.5},
         gammac/.style       = {magenta},
         gamma/.style        = {->,gammac,decorate,thick,
                                decoration={snake,amplitude=0.6mm,
                                segment length=3mm,post length=0.5mm}},
         eq/.style           = {scale=0.8}
        }
\tikzset{gd/.pic={code={%
\path (+0.1513414, +0.1494452) pic {neutron};
\path (-0.1308128, +0.2597228) pic {proton};
\path (-0.12820, -0.1626981) pic {neutron};
\path (+0.1562497, -0.1273654) pic {neutron};
\path (+0.1309994, -0.2506133) pic {proton};
\path (-0.34116, +0.3586174) pic {proton};}}}
\tikzset{alpha/.pic={code={%
\path (+0.1513414, +0.1494452) pic {neutron};
\path (-0.1308128, +0.2597228) pic {proton};
\path (+0.1562497, -0.1273654) pic {neutron};
\path (-0.34116, +0.3586174) pic {proton};}}}

\tikzset{alpha/.pic={code={%
\path (+0.15, +0.15) pic {neutron};
\path (-0.15, +0.15) pic {proton};
\path (+0.15, -0.15) pic {proton};
\path (-0.15, -0.15) pic {neutron};}}}

\tikzset{triton/.pic={code={%
\path (-0.15, +0.15) pic {proton};
\path (+0.15, -0.15) pic {proton};
\path (-0.15, -0.15) pic {neutron};}}}


\begin{tikzpicture}[
    source/.style={anchor=center,font=\Large},
  ]
  %
  % Coordinates
  %
  \coordinate (neutrino)     at (-2.0,  2.0);
  \coordinate (freeproton)   at ( 0.0,  1.0);
  \coordinate (gd)           at ( 4.0, -2.0); % (5.0,-2.0);
  \coordinate (alpha)      at ( 3.5, -0.5);
  \coordinate (triton)       at ( 5.5, -3.5);
  \coordinate (hydrogen)     at (-2.0, -2.0);
  \coordinate (electron)     at ( 5.0,  2.0);
  \coordinate (neutrino1)    at ($(freeproton)!2.5mm!(neutrino)$);


  \coordinate (toelectronA)  at ($(freeproton)!.22!30:(electron)$);
  \coordinate (toelectronB)  at ($(freeproton)!.4!-10:(electron)$);
  \coordinate (toelectronC)  at ($(freeproton)!.5!+5:(electron)$);
  \coordinate (toelectronD)  at ($(freeproton)!.9!-10:(electron)$);
  \coordinate (positron)     at ($(electron)!2mm!(toelectronD)$);
  \coordinate (annihilation) at ($(positron)!.5!(electron)$);

  \coordinate (togdA)        at ($(freeproton)!.22!-20:(gd)$);
  \coordinate (togdB)        at ($(freeproton)!.2!-5:(gd)$);
  \coordinate (togdC)        at ($(freeproton)!.5!+5:(gd)$);
  \coordinate (togdD)        at ($(freeproton)!.5!-35:(gd)$);
  \coordinate (togdE)        at ($(freeproton)!.4!-20:(gd)$);
  \coordinate (togdF)        at ($(freeproton)!.7!0:(gd)$);
  \coordinate (togdG)        at ($(freeproton)!.7!-20:(gd)$);
  \coordinate (togdH)        at ($(freeproton)!.9!-10:(gd)$);
  \coordinate (neutron_gd)   at ($(togdH)!.2mm!(togdG)$);
%  \coordinate (capture_gd)   at ($(neutron_gd)!.95!20:(togdH)$);
  \coordinate (capture_gd)   at ($(neutron_gd)!.395!0:(gd)$);
  
  \coordinate (agd)       at ($(alpha)!.04!0:(gd)$);  % place alpha?
  \coordinate (tgd)       at ($(triton)!.04!0:(gd)$);  % place triton?
  
  \coordinate (neutron) at (togdA);


  \coordinate (tohB)         at ($(togdA)!.3!+45:(hydrogen)$);
  \coordinate (tohC)         at ($(tohB)!.6!-30:(hydrogen)$);
  \coordinate (tohD)         at ($(tohC)!.4!-160:(hydrogen)$);
  \coordinate (tohE)         at ($(tohD)!.8!+40:(hydrogen)$);
  \coordinate (neutron_h)    at ($(hydrogen)!4mm!(tohE)$);
  \coordinate (capture_h)    at ($(neutron_h)!.5!(hydrogen)$);

  \coordinate (cut1)         at ($(togdA)+(0,-.4cm)$);
  \coordinate (cut2)         at ($(togdA)+(0,-5cm)$);

  %
  % IBD and positron part
  %
  \draw [->-=.5,particlepath] (neutrino) -- (freeproton) node [midway,above] {$\overline{\nu}_e$};
  \draw [line width=7pt,Gold] (freeproton) -- (toelectronA) -- (toelectronB) 
                             -- (toelectronC) -- (toelectronD) -- (electron);
  \draw [->-=.11,particlepath] (freeproton)
                            -- (toelectronA) node [midway,above,yshift=3pt] {$e^{\scriptscriptstyle+}$}
                            -- (toelectronB) -- (toelectronC) -- (toelectronD) -- (electron);

  \draw [->-=.5,particlepath] (freeproton) -- (togdA) node [midway,below,yshift=-3pt] {};
%  \path (neutron) pic   {neutron} ;%{$n$}; % moved to end
 % \node [white] at (togdA) {$n$};

  \draw node [interaction] at (annihilation) {};
  \draw [gamma] (annihilation) -- +(-30:15mm); %-10:15
  \draw [gamma] (annihilation) -- +(150:15mm);%170:15

  \path (freeproton) pic {pastproton};
  \node [opacity=0.6] at (freeproton) {$p$};
  \path (neutrino1)  pic {neutrino}; 
  \path (electron)   pic {electron};
  \node [scale=0.8] at (electron) {$e$}; % node [right,above,yshift=5mm] {$e^+e^-$}; % annihilation
  \path (positron)   pic {positron};
  \node [scale=0.7] at (positron) {$e^{\scriptscriptstyle+}$};

  \path (electron) node [gammac,eq,anchor=south,xshift=0.0cm,yshift=0.5cm]    {$1.022\text{ MeV}$}; %\sum E_\gamma=
  \path (freeproton) -- (positron) node [midway,eq,yshift=0.8cm,fill=Gold,rounded corners=5pt] 
  {$E_{e^+}\propto E_{\overline{\nu}}$};

  %
  % Gd capture part
  %
  \draw [particlepath] (togdA) -- (togdB) -- (togdC) -- (togdD) -- (togdE)
                               -- (togdF) -- (togdG) -- (togdH) -- (gd);
  \draw node [interaction] at  (capture_gd) {};
  %\draw [gamma] (gd) -- +(-100:30mm);
  %\draw [gamma] (gd) -- +( -60:30mm);
  %\draw [gamma] (gd) -- +( -30:30mm);
  %\draw [gamma] (gd) -- +(  30:30mm);
  %\draw [gamma] (gd) -- +( 120:30mm);
  
  \draw [heavyparticlepath] (agd) -- (gd);
  \path (alpha) pic {alpha};  % draw the alpha
  \node [black,fill=white,text opacity=1, fill opacity = 0.4]  at (alpha)       {\Large$\boldmath\alpha$};
  
  \draw [heavyparticlepath] (tgd) -- (gd);
  \path (triton) pic {triton};    % draw the triton
  \node [black,fill=white,text opacity=1, fill opacity = 0.4] at (triton)       {\Large t};

  \path (gd) pic {gd};
  \node [black,fill=white,text opacity=1,fill opacity=0.46,rounded corners=5pt] at (gd)
      {${}^{6}_3$Li};%  {$^{157}_{\phantom{1}64}$Gd};
  \path (neutron_gd) pic {neutron};
  \node [white] at (neutron_gd) {$n$};
 \path (gd)       node [Green!80!black,eq,anchor=south,xshift=1.850cm,yshift=1.05cm]    
 			{\large$Q(n,{}^6\text{Li}) = 4.78\  \text{MeV}$}; %\sum E_\gamma=
 \path (gd)       node [Green!80!black,eq,anchor=south,xshift=1.50cm,yshift=.5cm]    
 			{\large$\text{E}_{ee}\approx0.5\text{ MeV}$}; %\sum E_\gamma=

  \path (capture_gd) node [eq,anchor=north,xshift=-0.2cm,yshift=-0.9cm] {$t_\text{cap}\approx40\text{ $\mu$s}$};

  %
  % H capture part
  %
  \draw [particlepath] (togdA) -- (tohB) -- (tohC) -- (tohD) -- (tohE) -- (hydrogen);
  \draw node [interaction] at (capture_h) {};
\draw [gamma] (capture_h) -- +(-30:15mm);   % 2.2 MeV should be contained in LS

  \path (hydrogen) pic {proton};
  \node at (hydrogen) {$p$};
  \path (neutron_h) pic {neutron};
  \node [white] at (neutron_h) {$n$};
  
  %\path (capture_h) node [gammac,eq,anchor=south,xshift=0.0cm,yshift=+0.75cm]  {$2.2\text{ MeV}$}; %\sum E_\gamma=
  %\path (capture_h) node [eq,anchor=north,xshift=0.0cm,yshift=-0.9cm] {$t_\text{cap}\approx200\text{ $\mu$s}$};

  \path (capture_h) node [gammac,eq,anchor=north,xshift=0.0cm,yshift=-0.9cm]  {$2.2\text{ MeV}$}; %\sum E_\gamma=
 % \path (capture_h) node [eq,anchor=south,xshift=0.0cm,yshift=+0.75cm] {$t_\text{cap}\approx200\text{ $\mu$s}$};


%
% draw neutron from proton last, so it overwrites dashed lines
  \path (neutron) pic   {neutron} ;%{$n$};
  \node [white] at (togdA) {$n$};


  %
  % Misc
  %
  \draw [blue!50,dashed,thick] (cut1) -- (cut2);
  \draw node (labelngd) [anchor=north west,yshift=5.5mm] at (cut2) {{n${}^6$Li {\footnotesize($\sim\!80\%$ of captures)}}};
  \draw node (labelnh)  [anchor=north east,yshift=5mm] at (cut2) {{\footnotesize($\sim\!20\%$)} nH};
  
  
  \draw [blue!80!white, fill, opacity=0.1] (+2.5,-2) circle [radius=5.2] ; % draw a big circle
  \draw node (labelngd) [anchor=north west, yshift=-50mm,xshift=-10mm, black] {\Large\bf $\mathbf{}^{6}\text{Li}$-loaded liquid scintillator}; % label circle
\end{tikzpicture}

\end{document}

% vim: textwidth=0
